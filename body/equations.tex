% !Mode:: "TeX:UTF-8" 

\BiChapter{数学公式的输入方法}{Input methods of equations}
\BiSection{研究生院的公式规范}{Equations typesetting standard from graduate school}
论文中的公式应另起行,原则上应居中书写,与周围文字留有足够的空间区分开。
若公式前有文字(如“解”、“假定”等),文字空两格写,公式仍居中写。公式末不加标点。

公式应标注序号,并将序号置于括号内。 公式序号按章编排,如第~1~章第一个公式序号为“(1-1)”。公式的序号右端对齐。

公式较长时最好在等号“=”处转行,如难实现,则可在~$+$、$-$、$\times$、$\div$~运算符号处转行,转行时运算符号仅书写于转行式前,不重复书写。

文中引用公式时,一般用“见式~(1-1)”或“由公式~(1-1)”。

公式中用斜线表示“除”的关系时应采用括号,以免含糊不清,如~$a/(b\cos x)$。通常“乘”的关系在前,如~$a\cos x/b$而不写成~$(a/b)\cos x$。

不能用文字形式表示等式,如:$\textnormal{刚度}=\frac{{\textnormal{受力}}}{{\textnormal{受力方向的位移}}}$。

\textbf{对于数学公式的输入方法,网络上有一个比较全面权威的文档~\href{http://tug.ctan.org/cgi-bin/ctanPackageInformation.py?id=voss-mathmode}{Math mode}~请大家事先大概浏览一下。下面将对学位论文中主要用到的数学公式排版形式进行阐述。}

\BiSection{生成~\LaTeX~数学公式的两种方法}{Two methods of generating \LaTeX equations}
对于先前没有接触过~\LaTeX~的人来说,编写~\LaTeX~数学公式是一件很繁琐的事,尤其是对复杂的数学公式来说,更可以说是一件难以完成的任务。
实际上,生成~\LaTeX~数学公式有两种较为简便的方法,一种是基于~MathType~数学公式编辑器的方法,另一种是基于~MATLAB~商业数学软件的方法,
下面将分别对这两种数学公式的生成方法作一下简单介绍。
\BiSubsection{基于~MathType~软件的数学公式生成方法}{Generating method of equations based on MathType}
MathType~是一款功能强大的数学公式编辑器软件,能够用来在文本环境中插入~Windows OLE~图形格式的复杂数学公式,所以应用比较普遍。但此软件只有~30~天的试用期,之后若再继续使用则需要付费购买才行。网络上有很多破解版的~MathType~软件可供下载免费使用,
笔者推荐下载安装版本号在~6.5~之上的中文破解版。

在安装好~MathType~之后,若在输入窗口中编写数学公式,复制到剪贴板上的仍然是图形格式的对象。
若希望得到可插入到~\LaTeX~编辑器中的文本格式对象,则需要对~MathType~软件做一下简单的设置:在~MathType~最上排的按钮中依次选择“参数选项
$\to$转换”,在弹出的对话窗中选中“转换到其它语言(文字):”,在转换下拉框中选择“Tex~--~--~LaTeX 2.09 and later”,并将对话框最下方的两个复选框全部勾掉,点击确定,这样,再从输入窗口中复制出来的对象就是文本格式的了,就可以直接将其粘贴到~\LaTeX~
编辑器中了。按照这种方法生成的数学公式两端分别有标记\verb|\[|和标记\verb|\]|,在这两个标记之间才是真正的数学公式代码。

若希望从~MathType~输入窗口中复制出来的对象为图形格式,则只需再选中“公示对象(Windows OLE~图形)”即可。

\BiSubsection{基于~MATLAB~软件的数学公式生成方法}{Generating method of equations based on Matlab}
MATLAB~是矩阵实验室(Matrix Laboratory)的简称,是美国~MathWorks~公司出品的商业数学软件。它是当今科研领域最常用的应用软件之一,
具有强大的矩阵计算、符号运算和数据可视化功能,是一种简单易用、可扩展的系统开发环境和平台。

MATLAB~中提供了一个~latex~函数,它可将符号表达式转化为~\LaTeX~数学公式的形式。其语法形式为~latex(s),其中,~s~为符号表达式,
之后再将~latex~函数的运算结果直接粘贴到~\LaTeX~编辑器中。从~\LaTeX~数学公式中可以发现,其中可能包含如下符号组合:
\begin{verbatim*}
\qquad=两个空铅(quad)宽度
\quad=一个空铅宽度
\;=5/18空铅宽度
\:=4/18空铅宽度
\,=3/18空铅宽度
\!=-3/18空铅宽度
\ =一个空格
\end{verbatim*}
所以最好将上述符号组合从数学公式中删除,从而使数学公式显得匀称美观。

对于~word~等软件的使用者来说,在我们通过~MATLAB~运算得到符号表达式形式的运算结果时,在~word~中插入运算结果需要借助于~MathType~软件,
通过在~MathType~中输入和~MATLAB~运算结果相对应的数学表达形式,之后再将~MathType~数学表达式转换为图形格式粘贴到~word~中。实际上,
也可以将~MATLAB~中采用~latex~函数运行的结果直接粘贴到~MathType~中,再继续上述步骤,这样可以大大节省输入公式所需要的时间。
此方法在~MathType~6.5c~上验证通过,若您粘入到~MathType~中的仍然为从~MATLAB~中导入的代码,请您更新~MathType~软件。

\BiSection{数学字体}{Math fonts}
在数学模式下,常用的数学字体命令有如下几种:
\begin{verbatim}
\mathnormal或无命令 用数学字体打印文本;
\mathit             用斜体(\itshape)打印文本;
\mathbf             用粗体(\bfseries)打印文本;
\mathrm             用罗马体(\rmfamily)打印文本;
\mathsf             用无衬线字体(\sffamily)打印文本;
\mathtt             用打印机字体(\ttfamily)打印文本;
\mathcal            用书写体打印文本;
\end{verbatim}
在学位论文撰写中,只需要用到上面提到的~\verb|\mathit|、\verb|\mathbf|~和~\verb|\mathrm|~命令。若要得到~Times New Roman~的数学字体,则需要调用~txfonts~宏包(此宏包实际上采用的是~Nimbus Roman No9 L~字体,
它是开源系统中使用的免费字体,其字符字体与~Times New Roman~字体几乎完全相同);若要得到粗体数学字体,则需要调用~bm~宏包。表~\ref{table:fonts}~中分别列出了得到阿拉伯数字、拉丁字母和希腊字母
各种数学字体的命令。
\begin{table}[htbp]
\bicaption[table:fonts]{}{常用数学字体命令一览}{Table$\!$}{Summary of common commands for setting math fonts}
\vspace{0.5em}\centering\wuhao
\begin{tabular}{llll}
\toprule
 & 阿拉伯数字\&大写希腊字母 & 大小写拉丁字母 & 小写希腊字母  \\
\midrule
斜体 & \verb|\mathit{}| & \verb|无命令| & \verb|无命令|\\
粗斜体 & \verb|\bm{\mathit{}}| & \verb|\bm{}| & \verb|\bm{}|\\
直立体 & \verb|无命令| & \verb|\mathrm{}| & \verb|字母后加up|\\
粗体 & \verb|\mathbf{}或\bm{}| & \verb|\mathbf{}| & \verb|\bm{字母后加up}|\\
\bottomrule
\end{tabular}
\end{table}

\noindent 下面列出了一些应采用直立数学字体的数学常数和数学符号。

\vspace{-0.5em}\begin{center}\begin{tabularx}{0.7\textwidth}{XX}
$\mathrm{d}$、 $\mathrm{D}$、 $\mathrm{p}$~———微分算子 & $\mathrm{e}$~———自然对数之底数\\
$\mathrm{i}$、 $\mathrm{j}$~———虚数单位 & $\piup$———圆周率\\
\end{tabularx}\end{center}

\BiSection{行内公式}{Inline mode equations}
出现在正文一行之内的公式称为行内公式,例如~$f(x)=\int_{a}^{b}\frac{\sin{x}}{x}\mathrm{d}x$。对于非矩阵和非多行形式的行内公式,一般不会使得行距发生变化,而~word~等软件却会根据行内公式的竖直距离而自动调节行距,如图~\ref{hangju}~所示。
\begin{figure}[htbp]
\centering
\subfigure{\label{latex}}\addtocounter{subfigure}{-2}
\subfigure[Inline mode equation derived from \LaTeX system]{\subfigure[由~\LaTeX~系统生成的行内公式]
          {\fbox{\includegraphics[width=0.55\textwidth]{latex}}}}
\subfigure{\label{word}}\addtocounter{subfigure}{-2}
\subfigure[Inline mode equation displayed as .doc format file derived from word software]{\subfigure[由~word软件生成的~.doc~格式行内公式]
          {\fbox{\includegraphics[width=0.55\textwidth]{word}}}}
\subfigure{\label{pdf}}\addtocounter{subfigure}{-2}
\subfigure[Inline mode equation displayed as .pdf format file derived from word software]{\subfigure[由~word软件生成的~.pdf~格式行内公式]
          {\fbox{\includegraphics[width=0.55\textwidth]{pdf}}}}
\bicaption[hangju]{}{由~\LaTeX~和~word~生成的~3~种行内公式屏显效果}{Fig.$\!$}{Three kinds of inline mode equation displayed effects derived from \LaTeX and word}
\vspace{-1em}
\end{figure}
这三幅图分别为~\LaTeX~和~word~生成的行内公式屏显效果,从图中可看出,在~\LaTeX~文本含有公式的行内,在正文与公式之间对接工整,行距不变;而在~word~文本含有公式的行内,在正文与公式之间对接不齐,行距变大。因此从这一点来说,
\LaTeX~系统在数学公式的排版上具有很大优势。

\LaTeX~提供的行内公式最简单、最有效的方法是采用~\TeX~本来的标记———开始和结束标记都写作~\$,例如本段开始的例子可由下面的输入得到。
\verb|$f(x)=\int_{a}^{b}\frac{\sin{x}}{x}\mathrm{d}x$|

\BiSection{行间公式}{Displaymath mode equations}
位于两行之间的公式称为行间公式,每个公式都是一个单独的段落,例如
\[\int_a^b{f\left(x\right)\mathrm{d}x}=\lim_{\left\|\Delta{x_i}\right\|\to 0}\sum_i{f\left(\xi_i\right)\Delta{x_i}}\]
除人工编号外,\LaTeX~各种类型行间公式的标记见表~\ref{eqtag}。
\begin{table}[htbp]
\bicaption[eqtag]{}{各种类型行间公式的标记}{Table$\!$}{Tags for several kinds of displaymath mode equations}
\vspace{0.5em}\centering\wuhao
\begin{tabularx}{0.85\textwidth}{cXX}
\toprule
& 无编号 & 自动编号\\\midrule
单行公式 & \verb|\begin{displaymath}...... \end{displaymath}|~或~\verb|\[...\]| & \verb|\begin{equation} ...... \end{equation}|\\
多行公式 & \verb|\begin{eqnarray*} ...... \end{eqnarray*}| & \verb|\begin{eqnarray} ...... \end{eqnarray}|\\
\bottomrule
\end{tabularx}
\end{table}
另外,在自动编号的某行公式行尾添加标签~\verb|\nonumber|,可将该行转换为无编号形式。

行间多行公式需采用~\verb|eqnarray|~或~\verb|eqnarray*|~环境,它默认是一个列格式为~\verb|rcl|~的~3~列矩阵,并且中间列的字号要小一些,因此通常只将需要对齐的运算符号(通常为等号“=”)置于中间列。

\BiSection{可自动调整大小的定界符}{Delimiters with automatic adjustable sizes}
若在左右两个定界符之前分别添加命令~\verb|\left|~和~\verb|\right|,则定界符可根据所包围公式大小自动调整其尺寸,这可从式(\ref{nodelimiter})和式(\ref{delimiter})中看出。
\begin{equation}\label{nodelimiter}
(\sum_{k=\frac12}^{N^2})
\end{equation}
\begin{equation}\label{delimiter}
\left(\sum_{k=\frac12}^{N^2}\right)
\end{equation}
式(\ref{nodelimiter})和式(\ref{delimiter})是在~\LaTeX~中分别输入如下代码得到的。
\begin{verbatim}
(\sum_{k=\frac12}^{N^2})
\left(\sum_{k=\frac12}^{N^2}\right)
\end{verbatim}
\verb|\left|~和~\verb|\right|~总是成对出现的,若只需在公式一侧有可自动调整大小的定界符,则只要用“.”代替另一侧那个无需打印出来的定界符即可。

若想获得关于此部分内容的更多信息,可参见~\href{http://tug.ctan.org/cgi-bin/ctanPackageInformation.py?id=voss-mathmode}{Math mode}~文档的第~8~章“Brackets, braces and parentheses”。

\BiSection{数学重音符号}{Accents in math mode}
数学重音符号通常用来区分同一字母表示的不同变量,输入方法如下(需要调用~\verb|amsmath|~宏包):

\vspace{0.5em}\noindent\wuhao\begin{tabularx}{\textwidth}{Xc|Xc|Xc}
 \verb|\acute| & $\acute{a}$ & \verb|\mathring| & $\mathring{a}$ & \verb|\underbrace| & $\underbrace{a}$ \\
 \verb|\bar| & $\bar{a}$ & \verb|\overbrace| & $\overbrace{a}$ & \verb|\underleftarrow| & $\underleftarrow{a}$ \\
 \verb|\breve| & $\breve{a}$ & \verb|\overleftarrow| & $\overleftarrow{a}$ & \verb|\underleftrightarrow| & $\underleftrightarrow{a}$ \\
 \verb|\check| & $\check{a}$ & \verb|\overleftrightarrow| & $\overleftrightarrow{a}$ & \verb|\underline| & $\underline{a}$ \\
 \verb|\dddot| & $\dddot{a}$ & \verb|\overline| & $\overline{a}$ & \verb|\underrightarrow| & $\underrightarrow{a}$ \\
 \verb|\ddot| & $\ddot{a}$ & \verb|\overrightarrow| & $\overrightarrow{a}$ & \verb|\vec| & $\vec{a}$ \\
 \verb|\dot| & $\dot{a}$ & \verb|\tilde| & $\tilde{a}$ & \verb|\widehat| & $\widehat{a}$ \\
 \verb|\grave| & $\grave{a}$ & \verb|\underbar| & $\underbar{a}$ & \verb|\widetilde| & $\widetilde{a}$ \\
 \verb|\hat| & $\hat{a}$ 
\end{tabularx}\vspace{0.5em}
\xiaosi 当需要在字母~$i$~和~$j$~的上方添加重音符号时,为了去掉这两个字母顶上的小点,这两个字母应该分别改用~\verb|\imath|~和~\verb|\jmath|。

如果遇到某些符号不知道该采用什么命令能输出它时,则可通过~\href{http://detexify.kirelabs.org/classify.html}{Detexify$^2$~网站}来获取符号命令。若用鼠标左键在此网页的方框区域内画出你所要找的符号形状,则会在网页右方列出和你所画符号形状相近的~5~个符号及其相对应的~\LaTeX~输入命令。若所列出的符号中不包括你所要找的符号,还可通过点击“Select from the complete list!”的链接以得分从低到高的顺序列出所有符号及其相对应的~\LaTeX~输入命令。

最后,笔者建议大家还是要以~\href{http://tug.ctan.org/cgi-bin/ctanPackageInformation.py?id=voss-mathmode}{Math mode}~这篇~pdf~文档作为主要参考。若要获得最为标准、美观的数学公式排版形式,可以查查文档中是否有和你所要的排版形式相同或相近的代码段,通过修改代码段以获得你所要的数学公式排版形式。









