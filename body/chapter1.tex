% !Mode:: "TeX:UTF-8"
% 这个是为了WinEdt设置的,它的默认不是UTF8.
% !TeX root = ../main.tex
% 若xelatex编译非UTF8文件,需在每个文件中指定字符编码;
% main.tex中手动制定了\atemp和\usewhat参数;
\ifx\atempxetex\usewhat
%\XeTeXinputencoding "gbk"
\fi

\defaultfont

\titleformat{\chapter}[hang]{\xiaosan\bf\raggedright\hei\sf\boldmath}{\xiaoer\chaptertitlename}{18pt}{\xiaosan}
\titlespacing{\chapter}{0pt}{8pt}{16pt}

\makeatletter
\newskip\@footindent
\@footindent=1em

\renewcommand\footnoterule{\kern-3\p@ \hrule width 0.4\columnwidth \kern 2.6\p@}
\@addtoreset{footnote}{page}

\long\def\@makefntext#1{\@setpar{\@@par\@tempdima \hsize
\advance\@tempdima-\@footindent
\parshape \@ne \@footindent \@tempdima}\par
\noindent \hbox to \z@{\hss\@thefnmark\hspace{0.5em}}#1}

\renewcommand\thefootnote{\pinumber{\arabic{footnote}}}
\def\@makefnmark{\hbox{\textsuperscript{\@thefnmark}}}

\newcommand\pinumber[1]{\ifcase#1 \or \ding{172}\or \ding{173}\or
  \ding{174}\or \ding{175}\or \ding{176}\or \ding{177}%
  \or \ding{178}\or \ding{179}\or \ding{180}\or \ding{181}\else *\fi\relax}
\makeatother
%以上从\makeatletter到\makeatother为重定义脚注编号,使之带圆圈

\BiChapter{绪论}{Introduction}
\label{cha1:Introduction}

\BiSection{研究背景}{Background} %此处的第二个{}内是用于填入英语标题,南理工没要求,所以空着,但不要删除它,下同。
\label{sec11:Background}

XXXXXXXXXXXXXXXXXXXXXXXXXXXXXXXXXXXXXXXX

XXXXXXXXXXXXXXXXxxxx

XXXXXXxx \cite{psychology_recognition_87}


