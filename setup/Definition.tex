% !Mode:: "TeX:UTF-8"
% 若xelatex编译非UTF8文件,需在每个文件中指定字符编码;
% main.tex中手动制定了\atemp和\usewhat参数;
\ifx\atempxetex\usewhat
%\XeTeXinputencoding "gbk"
\fi

%%%%%%%%%%%%%%%%%%%%%%%%%%%%%%%%%%%%%%%%%%%%%%%%%%%%%%%%%%%
% 重定义字体命令
% 注意win2000,没有 simsun, 最好到网上找一个
% 一些字体是office2000 带的
%%%%%%%%%%%%%%%%%%%%%%%%%%%%%%%%%%%%%%%%%%%%%%%%%%%%%%%%%%%
\ifx\atempxetex\usewhat %xelatex调用系统字体;
\setmainfont{Times New Roman}   % 正文中的英文 采用 Times New Roman 字体;
\setsansfont{SimHei}   % 正文中的英文 采用 Times New Roman 字体;
\setCJKmainfont{SimSun}
\setCJKsansfont{SimHei}
\setCJKmonofont{LiSu}
\setCJKfamilyfont{song}{SimSun}
\setCJKfamilyfont{hei}{SimHei}
\setCJKfamilyfont{fs}{FangSong}
\setCJKfamilyfont{kai}{KaiTi}
\setCJKfamilyfont{li}{LiSu}
\fi
\newcommand{\song}{\CJKfamily{song}}    % 宋体   (Windows自带simsun.ttf)
\newcommand{\fs}{\CJKfamily{fs}}        % 仿宋体 (Windows自带simfs.ttf)
\newcommand{\kai}{\CJKfamily{nkai}}      % 楷体   (Windows自带simkai.ttf)
\newcommand{\hei}{\CJKfamily{hei}}      % 黑体   (Windows自带simhei.ttf)
\newcommand{\li}{\CJKfamily{li}}        % 隶书   (Windows自带simli.ttf)

%%%%%%%%%%%%%%%%%%%%%%%%%%%%%%%%%%%%%%%%%%%%%%%%%%%%%%%%%%%
% 重定义字号命令
%%%%%%%%%%%%%%%%%%%%%%%%%%%%%%%%%%%%%%%%%%%%%%%%%%%%%%%%%%%
\newcommand{\yihao}{\fontsize{26pt}{26pt}\selectfont}       % 一号, 1.倍行距
\newcommand{\erhao}{\fontsize{22pt}{22pt}\selectfont}       % 二号, 1.倍行距
\newcommand{\xiaoer}{\fontsize{18pt}{18pt}\selectfont}      % 小二, 单倍行距
\newcommand{\sanhao}{\fontsize{16pt}{16pt}\selectfont}      % 三号, 1.倍行距
\newcommand{\xiaosan}{\fontsize{15pt}{15pt}\selectfont}     % 小三, 1.倍行距
\newcommand{\sihao}{\fontsize{14pt}{14pt}\selectfont}       % 四号, 1.0倍行距
\newcommand{\banxiaosi}{\fontsize{13pt}{13pt}\selectfont}   % 半小四, 1.0倍行距
\newcommand{\xiaosi}{\fontsize{12pt}{12pt}\selectfont}      % 小四, 1.倍行距
\newcommand{\dawuhao}{\fontsize{11.5pt}{11.5pt}\selectfont} % 大五号, 单倍行距
\newcommand{\wuhao}{\fontsize{10.5pt}{10.5pt}\selectfont}   % 五号, 单倍行距
\newcommand{\xiaowu}{\fontsize{9.5pt}{9.5pt}\selectfont}    % 五号, 单倍行距
\newcommand{\banbanxiaosi}{\fontsize{12pt}{12pt}\selectfont}% 半半小四, 1.0倍行距


%避免宏包 hyperref 和 arydshln 不兼容带来的目录链接失效的问题。
\def\temp{\relax}
\let\temp\addcontentsline
\gdef\addcontentsline{\phantomsection\temp}
\newcommand*{\subfigencaptionlist}{} % 子图形加入目录时用

\makeatletter
\gdef\hitempty{}

\newcommand{\mr}[1]{\mathrm{#1}} %定义新命令,用\mr来代替\mathrm
\def \ReferenceEName {References} %%定义参考文献的标题
\def \ReferenceCName {参考文献}

%定义图表章节双标题命令
\newcommand{\figenname}{Fig.}
\newcommand{\listfigenname}{List of Figures}
\newfloatlist[chapter]{figen}{fen}{\listfigenname}{\figenname}
\newfixedcaption{\figencaption}{figen}
\renewcommand{\thefigen}{\thechapter.\arabic{figure}}
\renewcommand{\@cftmakefentitle}{\chapter*{\listfigenname\@mkboth{\bfseries\listfigenname}{\bfseries\listfigenname}}}


\newcommand{\FigureBiCaption}[2]
{\renewcommand{\figurename}{图}
\caption{\protect\setlength{\baselineskip}{1.5em}#1} %\protect\setlength{\baselinestretch}{1.3}\selectfont
\vspace{-1.3ex}%-0.5ex
\figencaption{\protect\setlength{\baselineskip}{1.5em}#2}%
\vspace{-3.4mm}
%%
%%其子图形加入目录
\def\hittemp{}
 \@for \hittemp:=\subfigencaptionlist \do {%
        \ifx \hitempty\hittemp\relax \else
          \addcontentsline{fen}{subfigen}{\protect\numberline\hittemp}
        \fi}
 \gdef\subfigencaptionlist{}
}


\setcounter{fendepth}{2} %英文图形目录的深度 1(只有一级目录) 2(有两级目录)
\setcounter{lofdepth}{2} %中文图形目录的深度 1(只有一级目录) 2(有两级目录)
\renewcommand*{\l@subfigure}{\@dottedxxxline{\ext@subfigure}{2}{3.8em}{1.5em}} %中文图形目录 subfigure
\gdef\l@subfigen{\@dottedtocline{2}{3.8em}{1.5em}}%英文图形目录 latex
\newif\ifsubfigtoc
\ifnum \tw@ > \@nameuse{c@fendepth} \subfigtocfalse \else \subfigtoctrue \fi
\newbox\tempbox
\newcommand{\SubfigEnCaption}[1]
{\ifsubfigtoc
    %加入目录这个动作,一定要在 父图 之后,所在先暂存在 subfigencaptionlist
    \protected@xdef\subfigencaptionlist{\subfigencaptionlist,%
        {{\thesubfigure}\protect\ignorespaces{#1}}}
\else
    \relax
\fi
%产生caption
\vspace{1pt}
\sbox{\tempbox}{\thesubfigure\hskip\subfiglabelskip #1}%
\ifthenelse{\lengthtest{\wd\tempbox > \linewidth}}%
{\\[-20pt]\parbox[t]{\linewidth}{\flushleft\noindent\wuhao\selectfont\thesubfigure\hskip\subfiglabelskip \centering#1\hangafter=1\hangindent=15pt}}%
{\\\centerline{\wuhao\selectfont\thesubfigure\hskip\subfiglabelskip #1}}
}

%\newcommand{\SubfigureCaption}[2]  % Two Parameters, the first one is the width of the subfigure,
%{
%\addtocounter{subfigure}{-1}       % the second one is the caption of the subfigure
%\vspace{-2ex}
%\subfigure[#2]{\rule{#1}{0pt}}
%}

\newcommand{\tblenname}{Table} %define tbl instead of table
\newcommand{\listtblenname}{List of Tables}
\newfloatlist[chapter]{tblen}{ten}{\listtblenname}{\tblenname}
\newfixedcaption{\tblencaption}{tblen}
\renewcommand{\thetblen}{\thechapter.\arabic{table}}% 将tblen换成table,因为table和tablen编号一致,而tablen在\longbitoccaption定义中无效。
\renewcommand{\@cftmaketentitle}{\chapter*{\listtblenname\@mkboth{\bfseries\listtblenname}{\bfseries\listtblenname}}}


\newcommand{\TableBiCaption}[2]
{
\renewcommand{\tablename}{表}
\caption{\protect\setlength{\baselineskip}{1.5em}#1}
\vspace{-2ex}
\tblencaption{\protect\setlength{\baselineskip}{1.5em}#2}
\vspace{1ex}
}

%%%% 长表格的caption在中英文表格目录中正常显示
\def\@cont@LT@LTBiToeCaption#1[#2]#3{%
  \LT@makecaption#1\fnum@table{#3}%
  \def\@tempa{#2}%
  \ifx\@tempa\@empty\else
    {\let\\\space
      %\phantomsection
      \addcontentsline{ten}{tblen}{\protect\numberline{\thetable}{#2}}}%%\addcontentsline{lot}{table}{\protect\numberline{}{#2}}}%
  \fi}
\def\LT@c@ption#1[#2]#3{%
  \LT@makecaption#1\fnum@table{#3}%
  \def\@tempa{#2}%
  \ifx\@tempa\@empty\else
     {\let\\\space
     %\phantomsection
     \addcontentsline{lot}{table}{\protect\numberline{\thetable}{#2}}}%
  \fi}
\let\@cont@oldLT@c@ption\LT@c@ption
\newcommand*{\LTBiTocCaption}[5]{
  \@if@contemptyarg{#1}{\caption{#2}}{\caption[#1]{#2}}%
  \global\let\@cont@oldtablename\tablename
  \gdef\tablename{Table} %#3
  \global\let\LT@c@ption\@cont@LT@LTBiToeCaption
  \\
  \@if@contemptyarg{#4}{\caption{#5}}{\caption[#4]{#5}}%
  \global\let\tablename\@cont@oldtablename
  \global\let\LT@c@ption\@cont@oldLT@c@ption}

\renewcommand{\cfttblendotsep}{1} %自定义图表目录中的点间距大小
\renewcommand{\cftfigendotsep}{1}

%\renewcommand{\tablename}{表}  %jdg提供的一种方法,英文长表会添加到中文目录中去,上面定义
%\newcommand{\LTBiCaption}[2]   %\bicaptiontwotoc 主要就是解决这个问题。
%{%
%\caption{#1} \gdef\tablename{Table}
%\\ %[-3.5ex]
%\caption{#2}
%\gdef\tablename{表}\\ %[-1.5ex]
%}

%%%---公式中符号描述----start----
%\begin{formulasymb}{式中}{-3pt}%-3pt,-20pt调与上方的间距。
%  \fdesfirst{第一标签}{控制控制控制控制控制}
%  \fdes{其他标签}{控制控制控制控制控制}
%\end{formulasymb}
\newenvironment{formulades}[1]%
{\noindent\begin{list}{}{%
\setlength\topsep{0pt}
\settowidth{\labelwidth}{#1}
\setlength{\labelsep}{1mm}
\setlength{\leftmargin}{\labelwidth+\labelsep}
}}{\end{list}}
\newenvironment{formulasymb}[2]%-\!-\!-\!-
{\vspace*{#2}\newcommand{\fdesfirst}[2]%
{\begin{formulades}{#1\hspace*{26pt}##1~\cdash}\item[#1\hspace*{26pt}##1~\cdash]{##2}\end{formulades}\vspace*{-21pt}}%自己调距
\newcommand{\fdes}[2]{\begin{formulades}{#1\hspace{26pt}##1~\cdash}\item[##1~\cdash]{##2}\end{formulades}\vspace*{-21pt}}}%自己调距
{\vspace{21pt}\relax}%21pt调距
%%----公式中符号描述----end-----

%% ---- 左对齐的公式 start-----   \begin{flualign} a=c \end{flualign}
\newenvironment{flualign}{%
    \@fleqntrue
    \@mathmargin = -1sp
    \@mathmargin\leftmargini minus\leftmargini
    \let\mathindent=\@mathmargin
  \start@align\@ne\st@rredfalse\m@ne
}{%
  \math@cr \black@\totwidth@
  \egroup
  \ifingather@
    \restorealignstate@
    \egroup
    \nonumber
    \ifnum0=`{\fi\iffalse}\fi
  \else
    $$%
  \fi
  \ignorespacesafterend
  \@fleqnfalse
}
%%  ---- 左对齐的公式  end----

%重新定义BiChapter命令,可实现标题手动换行,但不影响目录
\def\BiChapter{\relax\@ifnextchar [{\@BiChapter}{\@@BiChapter}}
\def\@BiChapter[#1]#2#3{\chapter[#1]{#2}
    \addcontentsline{toe}{chapter}{\bfseries Chapter \thechapter\hspace{0.5em} #3}}
\def\@@BiChapter#1#2{\chapter{#1}
    \addcontentsline{toe}{chapter}{\bfseries Chapter \thechapter\hspace{0.5em}{\boldmath #2}}}
%\newcommand{\BiChapter}[2]
%{
%    \chapter{#1}
%    \addcontentsline{toe}{chapter}{\bfseries Chapter \thechapter\hspace{0.5em} #2}
%}

\newcommand{\BiSection}[2]
{   \section{#1}
    \addcontentsline{toe}{section}{\protect\numberline{\csname thesection\endcsname}#2}
}

\newcommand{\BiSubsection}[2]
{    \subsection{#1}
    \addcontentsline{toe}{subsection}{\protect\numberline{\csname thesubsection\endcsname}#2}
}

\newcommand{\BiSubsubsection}[2]
{    \subsubsection{#1}
    \addcontentsline{toe}{subsubsection}{\protect\numberline{\csname thesubsubsection\endcsname}#2}
}

\newcommand{\BiAppendixChapter}[2] % 该附录命令适用于发表文章,简历等
{\phantomsection
\markboth{#1}{#1}%\markboth{\MakeUppercase{#1}}{\MakeUppercase{#1}}
\addcontentsline{toc}{chapter}{\hei #1}
\addcontentsline{toe}{chapter}{\bfseries #2}  \chapter*{#1}
}

\newcommand{\BiAppChapter}[2]    % 该附录命令适用于有章节的完整附录
{\phantomsection  \chapter{#1}   %\markboth{\MakeUppercase{#1}}{\MakeUppercase{#1}} %为了winedt中project tree中toc正确显示,不要挪到下一行;
%\addcontentsline{toc}{chapter}{\hei #1}
 \addcontentsline{toe}{chapter}{\bfseries Appendix \thechapter~~#2}
}

\renewcommand{\thefigure}{\arabic{chapter}.\arabic{figure}}%使图编号为 7.1 的格式 %\protect{~}
%\renewcommand\fnum@figure{\figurename\nobreakspace\thefigure\protect{~~~~~~~~~}} %

\renewcommand{\thesubfigure}{\alph{subfigure})}%使子图编号为 a)的格式
\renewcommand{\p@subfigure}{\thefigure(} %%使子图引用为 7.1(a) 的格式
%\renewcommand{\thesubfigure}{\alph{subfigure}}
%\renewcommand{\p@subfigure}{\thefigure} %%使子图引用为 7.1a 的格式
%\renewcommand{\@thesubfigure}{\thesubfigure)\hskip\subfiglabelskip}%使子图编号为 a)的格式

\renewcommand{\thetable}{\arabic{chapter}.\arabic{table}}%%使表编号为 7.1 的格式
\renewcommand{\theequation}{\arabic{chapter}.\arabic{equation}}%%使公式编号为 7.1 的格式

% 设置算法标题形式,由“Algorithm 2.1:算法标题” 改为“算法 2.1 算法标题”
\renewcommand{\algorithmcfname}{算法}
\setlength\AlCapSkip{1.2ex}
\SetAlgoSkip{1pt}
\renewcommand{\algocf@captiontext}[2]{#1\algocf@typo ~ \AlCapFnt{}#2} % text of caption
\expandafter\ifx\csname algocf@within\endcsname\relax% if \algocf@within doesn't exist
\renewcommand\thealgocf{\@arabic\c@algocf} % and the way it is printed
\else%                                    else
\renewcommand\thealgocf{\csname the\algocf@within\endcsname.\@arabic\c@algocf}
\fi

\makeatother
%定义 学科 学位
\def \xuekeEngineering {Engineering}
\def \xuekeScience {Science}
\def \xuekeManagement {Management}
\def \xuekeArts {Arts}

\ifx \xueke \xuekeEngineering
\newcommand{\cxueke}{工学}
\newcommand{\exueke}{Engineering}
\fi

\ifx \xueke \xuekeScience
\newcommand{\cxueke}{理学}
\newcommand{\exueke}{Science}
\fi

\ifx \xueke \xuekeManagement
\newcommand{\cxueke}{管理学}
\newcommand{\exueke}{Management}
\fi

\ifx \xueke \xuekeArts
\newcommand{\cxueke}{文学}
\newcommand{\exueke}{Arts}
\fi

\newcommand{\cdash}{\mbox{—\!\!\!\!—\!\!\!\!—}}%输入中文破折号的命令
\newcommand{\dif}{\mathrm{d}}%在数学模式中输入微分dx

%将参考文献在文中的引用改成上标形式 --------by tina
%\newcommand{\upcite}[1]{\textsuperscript{\textsuperscript{\cite{#1}}}}
\newcommand{\upcite}[1]{\textsuperscript{\cite{#1}}} 