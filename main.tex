% !Mode:: "TeX:UTF-8"
% The main file for the thesis.
% main.tex
%

% 编译方式,可选的有xelatex, pdflatex, dvipdfmx, dvipspdf, yap
% 推荐使用xelatex,默认UTF8编码格式,对中文支持很好。
\def\usewhat{xelatex}

% 若用xelatex编译,默认用UTF8编码,非UTF8文件时,需在每个文件中指定字符编码;
% 本段无需改动,需放在输入中文之前;
\def\atempxetex{xelatex}\ifx\atempxetex\usewhat
% 若文件使用非utf8编码,请将注释去掉
% \XeTeXinputencoding "gbk"
\fi

% input "reference\reference.bib" %for winedt users
% 版本号: X.Y.Z, X: major, Y: minor, Z: patch
\def\version{1.0.0}     % 该变量仅用于模板文件的版本号控制

\def\xuewei{Doctor}   % 定义学位 博士
%\def \xuewei {Master}  % 硕士

\def\oneortwoside{twoside} %定义单双面打印,只对硕士学位论文有效;
%\def\oneortwoside{oneside} % 硕士单面打印

\def\xueke{Engineering} % 定义学科 工学
%\def\xueke{Science}      % 理学
%\def\xueke{Management}   % 管理学
%\def\xueke{Arts}         % 艺术学

% dicta 2009 definition.
\DeclareMathAlphabet{\mathpzc}{OT1}{pzc}{m}{it}
\def\calV{\mathcal{V}}
\def\calE{\mathcal{E}}
\def\calC{\mathcal{C}}
\def\calU{\mathcal{U}}
\def\bfX{\mathbf{X}}
\def\bfx{\mathbf{x}}
\def\bfY{\mathbf{Y}}
\def\bfy{\mathbf{y}}
\def\bfL{\mathbf{L}}
\def\bfE{\mathbf{E}}
\def\bfD{\mathbf{D}}
\def\bfW{\mathbf{W}}
\def\bfT{\mathpzc{T}}
\def\bfone{\mathbf{1}}
\def\overX{\overline{X}}
\def\vector{\mathrm{vec}}
\def\lsep{\!\!&}
\def\rsep{&\!\!}
% end of dicta 2009

\input{setup/type.tex}    % 硕博类型

%下面的book选项中可以使用 draft 选项,使插入的图形只显示外框,以加快预览速度。
\documentclass[12pt,a4paper,openany,\oneortwoside]{book}
\input{setup/package.tex} % 引用的宏包

% 论文包含的内容
\includeonly{
            appendix/Authorization,
            body/chapter1,
            body/chapter2,
            body/chapter3,
            body/chapter4,
            body/chapter5,
            body/conclusion,
            body/acknowledgements,
            %appendix/appA,
            appendix/publications,  % 编译此文件时 一定要配合一章内容 否则会报错 --- by Tina
%            appendix/Resume
            }
\graphicspath{{figures/}} % 定义所有的eps文件在 figures 子目录下

\begin{document}
\ifx\atempxetex\usewhat\else
\begin{CJK*}{UTF8}{song}
\fi

\input{setup/Definition} % 文本格式定义
\input{setup/format}

%%%%%%%%%%%%%%%%%%%%%%%%%%%%%%%%%%%%%%%%%%%%%%%%%%
% 正文部分
%%%%%%%%%%%%%%%%%%%%%%%%%%%%%%%%%%%%%%%%%%%%%%%%%%
\frontmatter
\sloppy % 解决中英文混排的断行问题,会加入间距,但不会影响断行
%\interlinepenalty -100000

%%% ****** 封面 目录 ******%%%
% !Mode:: "TeX:UTF-8"
% 若xelatex编译非UTF8文件,需在每个文件中指定字符编码;
% main.tex中手动制定了\atemp和\usewhat参数;
\ifx\atempxetex\usewhat
%\XeTeXinputencoding "gbk"
\fi

\newcommand{\chinesethesistitle}{xxxxxxxxxxxxxxxx} %授权书用,无需断行
\newcommand{\englishthesistitle}{xxxxxxxxxxxxxxxxg}
\newcommand{\chinesethesistime}{2013~年~6月}  %封面底部的日期中文形式
\newcommand{\dchinesethesistime}{2013~年~6月} %答辩日期中文形式:2011~年~~月
\newcommand{\englishthesistime}{June, 2013}   %封面底部的日期英文形式
%-----定义标签值-与format.tex文件中的定义相对应----------------
\lnatclassifiedindex{分类号}
\lsecretclassifiedindex{密级}
\linternatclassifiedindex{UDC}

\lsupervisor{指导教师}
\lname{姓名}
\lauthor{作\hspace*{2em}者}
\ldegree{申请学位级别}
\lsubject{专业名称}
\ldate{论文提交日期}
\lddate{论文答辩日期}
\lschool{学位授予单位和日期}
\lchairman{答辩委员会主席}
\lexpositor{评阅人}
\cabstract{摘\hspace*{2em}要}
\ckeywords{关键词}

%----定义标签对应选项的值---------------
\natclassifiedindex{TP309}
\secretclassifiedindex{公开}
\internatclassifiedindex{~681.324} %国际图书分类号

%\school{南京理工大学}
%\cdegree{\cxuewei}
\ctitle{\erhao\hei xxxxxxxxxxxxxxxx}  %封面用论文标题,自己可手动断行
\csubject{xxxxxxxxxxxxxxxx}                 %(~按二级学科填写~)
\cauthor{xxxx}            %作者姓名:XXX
\crole{教授}         %导师职称:教授
\csupervisor{XXXX} %导师名字:XXXX
%\cassosupervisor{某~~~~~~某~~~~教~~授}     %(~如无副导师可以不列此项~)
%\ccosupervisor{某~~某~~某~~~~教~~授~} %(~如无联合培养导师则不列此项~)
\cdate{\chinesethesistime}
\ddate{\dchinesethesistime}%定义答辩日期

\etitle{\englishthesistitle}
\edegree{\exuewei \ of \exueke}
\esubject{Pattern Recognition and Intelligence Systems}  %英文二级学科名
\eaffil{School of Computer Science and Technology}%英文单位 %换行用\newline,不要用\\
\eauthor{\textbf{XXXX}}                   %作者姓名 (英文):\textbf{Yantingt Lu}
\esupervisor{\textbf{XXXX}}       % 导师姓名 (英文):\textbf{Jing-Yu Yang}
%\ecosupervisor{Professor X}
%\eassosupervisor{Professor Y}
\edate{\englishthesistime}


\cabstract{
XXXXXXXXXXXXXXXXXXXXXXXXXXXXXXXXXXX

XXXXXXXXXXXXXXXXXXXXXXXXXXXXXXXXXXX

}

%关键词不能放在摘要页的外面
 \ckeywords{XXXXX,\ XXXXX,\ XXXXX, \ XXXXX, \ XXXXX, \ XXXXX, \ XXXXX}



\eabstract{
XXXXXXXXXXXXXXXXXXXXXXXXXXXXXXXXXXX

XXXXXXXXXXXXXXXXXXXXXXXXXXXXXXXXXXX

}

\ekeywords{XXXXX, \ XXXXX, \ XXXXX, \ XXXXX,  \ XXXXX,  \ XXXXX, \ XXXXX}

%主要符号说明
%\NotationList{
%%\vspace*{10pt}%可在主要符号说明与下面内容之间留空
%\begin{center}
%\begin{tabular}{lll}
%符号 & 名字 & 定义\\
%\hline
%$\mathcal {X}$,$\mathcal {Y}$,$\mathcal {Z}$ & 花体大写英文& 张量\\
%$\textbf{A}$, $\textbf{B}$, $\textbf{C}$ & 粗体大写英文& 矩阵 \\
%$\textbf{u}$, $\textbf{v}$, $\textbf{w}$ & 粗体小写英文& 向量 \\
%$l, m, n$ & 常规斜体小写英文& 数量 \\
%$L, M, N$ & 常规斜体大写英文& 数量指标的上界(如$m=1,\ldots,M$) \\
%$\alpha ,\beta ,\theta $ & 小写希腊字母 & 参数 \\
%$I_{n}$ & 专用符号符号 & $n\times n$单位矩阵 \\
%$1_{m\times n}$ & 专用符号 & $m\times n$全一矩阵 \\
%$\epsilon$ & 专用符号 & 误差 \\
%$\lambda $ & 专用符号 & 矩阵的特征值 \\
%$\otimes  $ & 专用符号 & Kronecker积 \\
%$\langle \bullet , \bullet \rangle$ & 专用符号 & 向量内积 \\
%\hline
%\end{tabular}
%\end{center}
%}
\makecover
\clearpage
 % 封面
 \renewcommand{\baselinestretch}{1}
 \fontsize{12pt}{12pt}\selectfont
 \clearpage{\pagestyle{empty}\cleardoublepage}
 \pdfbookmark[0]{目~~~~录}{mulu}
 \tableofcontents    % 中文目录

%%% ****** 图标索引,符号表 ******%%%
 \input{setup/figtab.tex}  %图表索引, 如果不需要图表索引,注释掉这一句即可;
 %\notation  %主要符号表
 \addtocontents{toc}{\protect\vskip1\baselineskip} % 中文目录增加空行

%%% ****** 清除页眉页脚 ******%%%
\ifxueweidoctor
  \clearpage{\pagestyle{empty}\cleardoublepage}   % 清除目录后面空页的页眉和页脚
\else%
  \ifoneortwoside\clearpage{\pagestyle{empty}\cleardoublepage}\fi  % 清除目录后面空页的页眉和页脚
\fi                                               %  第一章 是否右开

\mainmatter
\defaultfont % 对应于小四的标准字号是 12pt, 可以在正文中用此命令修改所需要字体的的大小

% !Mode:: "TeX:UTF-8"
% 这个是为了WinEdt设置的,它的默认不是UTF8.
% !TeX root = ../main.tex
% 若xelatex编译非UTF8文件,需在每个文件中指定字符编码;
% main.tex中手动制定了\atemp和\usewhat参数;
\ifx\atempxetex\usewhat
%\XeTeXinputencoding "gbk"
\fi

\defaultfont

\titleformat{\chapter}[hang]{\xiaosan\bf\raggedright\hei\sf\boldmath}{\xiaoer\chaptertitlename}{18pt}{\xiaosan}
\titlespacing{\chapter}{0pt}{8pt}{16pt}

\makeatletter
\newskip\@footindent
\@footindent=1em

\renewcommand\footnoterule{\kern-3\p@ \hrule width 0.4\columnwidth \kern 2.6\p@}
\@addtoreset{footnote}{page}

\long\def\@makefntext#1{\@setpar{\@@par\@tempdima \hsize
\advance\@tempdima-\@footindent
\parshape \@ne \@footindent \@tempdima}\par
\noindent \hbox to \z@{\hss\@thefnmark\hspace{0.5em}}#1}

\renewcommand\thefootnote{\pinumber{\arabic{footnote}}}
\def\@makefnmark{\hbox{\textsuperscript{\@thefnmark}}}

\newcommand\pinumber[1]{\ifcase#1 \or \ding{172}\or \ding{173}\or
  \ding{174}\or \ding{175}\or \ding{176}\or \ding{177}%
  \or \ding{178}\or \ding{179}\or \ding{180}\or \ding{181}\else *\fi\relax}
\makeatother
%以上从\makeatletter到\makeatother为重定义脚注编号,使之带圆圈

\BiChapter{绪论}{Introduction}
\label{cha1:Introduction}

\BiSection{研究背景}{Background} %此处的第二个{}内是用于填入英语标题,南理工没要求,所以空着,但不要删除它,下同。
\label{sec11:Background}

XXXXXXXXXXXXXXXXXXXXXXXXXXXXXXXXXXXXXXXX

XXXXXXXXXXXXXXXXxxxx

XXXXXXxx \cite{psychology_recognition_87}



\include{body/chapter2}
\include{body/chapter3}
% !Mode:: "TeX:UTF-8"
% !TeX root = ../main.tex
% 若xelatex编译非UTF8文件,需在每个文件中指定字符编码;
% main.tex中手动制定了\atemp和\usewhat参数;
\ifx\atempxetex\usewhat
%\XeTeXinputencoding "gbk"
\fi
\defaultfont



\BiChapter{XXXXXX}{XXXXXXXXXXXx}
\label{cha4}

\BiSection{引言}{Introduction}
\label{sec41}





% !Mode:: "TeX:UTF-8"
% !TeX root = ../main.tex
% 这个是为了WinEdt设置的,它的默认不是UTF8.

% 若xelatex编译非UTF8文件,需在每个文件中指定字符编码;
% main.tex中手动制定了\atemp和\usewhat参数;
\ifx\atempxetex\usewhat
%\XeTeXinputencoding "gbk"
\fi
\defaultfont

\BiChapter{XXXXx}{XXXXXXXXXXXXX }
\label{chp6}

\BiSection{引言}{Introduction}
\label{sec6}


\include{body/conclusion}   % 结束语
\include{body/acknowledgements}% 致谢
%%% ******** 参考文献***************%%%
\defaultfont
\ifx\atempxetex\usewhat
\bibliographystyle{chinesebst2005}
\else
\bibliographystyle{chinesebst}
\fi
\addcontentsline{toc}{chapter}{\hei \ReferenceCName}      % 参考文献加入到中文目录
\addcontentsline{toe}{chapter}{\bfseries \ReferenceEName} % 参考文献加入到英文目录
\addtolength{\bibsep}{-0.8 em}

%\nocite{*}  %如果不注释掉,会将bib文件中所有的引文都列出来,不管是否在正文中引用。
\bibliography{reference/reference}

%\addtocontents{fen}{\protect\vskip1\baselineskip}
%\addtocontents{ten}{\protect\vskip1\baselineskip}
%英文图形和表格索引里加入空白行,通常放在 % !Mode:: "TeX:UTF-8"
% 这个是为了WinEdt设置的,它的默认不是UTF8.
% !TeX root = ../main.tex
% 若xelatex编译非UTF8文件,需在每个文件中指定字符编码;
% main.tex中手动制定了\atemp和\usewhat参数;
\ifx\atempxetex\usewhat
%\XeTeXinputencoding "gbk"
\fi



%%%%%%%%%%%%%%%%%%%%%%%%%%%%%%%%%%%%%%%%%%%%%%%%%%%%%%%%
 \defaultfont
 \appendix

 % %%%%%%%%%%%%%%%%%%%%%%%%%%%%%%%%%%%%%%%%%%%%%%%%%%%%%%%%%
% 附录A之前。
%区分开正文和附录的图形和表格,一般没有这个必要。

%%% ******** 附件***************%%%
%%------------注意(womegaga): 下面三行是撰写每一章时,为避免不必要的编译,所以注释掉,最终论文需要去掉下面三行注释,并进行相应的核实和修改%%

% !Mode:: "TeX:UTF-8"
% 这个是为了WinEdt设置的,它的默认不是UTF8.
% !TeX root = ../main.tex
% 若xelatex编译非UTF8文件,需在每个文件中指定字符编码;
% main.tex中手动制定了\atemp和\usewhat参数;
\ifx\atempxetex\usewhat
%\XeTeXinputencoding "gbk"
\fi



%%%%%%%%%%%%%%%%%%%%%%%%%%%%%%%%%%%%%%%%%%%%%%%%%%%%%%%%
 \defaultfont
 \appendix

 % %%%%%%%%%%%%%%%%%%%%%%%%%%%%%%%%%%%%%%%%%%%%%%%%%%%%%%%%%
            % 附录A
% !Mode:: "TeX:UTF-8"
% 若xelatex编译非UTF8文件,需在每个文件中指定字符编码;
% main.tex中手动制定了\atemp和\usewhat参数;
\ifx\atempxetex\usewhat
%\XeTeXinputencoding "gbk"
\fi
\defaultfont
\BiAppendixChapter{附~~~~录}{Fulu}
%\BiAppendixChapter{攻读\cxuewei 学位期间发表的学术论文} {Papers Published in the Period of PH. D. Education}

\song\sihao\sf\bf{攻读博士期间完成的已发表的论文:}
\vspace{10pt}
\begin{publist}

\xiaosi \item XXXXXX

\vspace{10pt}
\item XXXXXXX


\vspace{10pt}
\item XXXXXx





\end{publist}



\vspace{100pt}
\song\sihao\sf\bf{作者在攻读博士学位期间参与的主要科研项目:}
\vspace{10pt}
\begin{publist}
\xiaosi \item XXXXXXX。
 \vspace{10pt}
 \item XXXXXX”。
\end{publist}
    % 所发文章
% !Mode:: "TeX:UTF-8"
% 若xelatex编译非UTF8文件,需在每个文件中指定字符编码;
% main.tex中手动制定了\atemp和\usewhat参数;
%\ifx\atempxetex\usewhat
%%\XeTeXinputencoding "gbk"
%\fi
%\defaultfont
%
%\BiAppendixChapter{个人简历}{Resume}
%
%{\hei 学习经历}
%\begin{publist}
%\item 2007~年~3~月--至今~~南京理工大学计算机科学与技术学院~~~攻读工学博士学位
%\item 2004~年~9~月~~2006~年~12~月~~~南京林业大学信息技术学院~~~获工学硕士学位
%\item 2000~年~9~月--2004~年~6~月~~~南京邮电学院应用数理系~~~获理学学士学位
%\end{publist}
%
%{\hei 科研工作}
%\begin{publist}
%\item  2007~年~x~月--xxxx~年~x~月 ~~~ xxxx项目~~~~(编号xxx-xxx-xxx) 
%\item  xxxx~年~x~月--xxxx~年~x~月 ~~~ xxxx项目~~~~(编号xxx-xxx-xxx)
%\item  xxxx~年~x~月--xxxx~年~x~月 ~~~ xxxx项目~~~~(编号xxx-xxx-xxx)
%\item  xxxx~年~x~月--xxxx~年~x~月 ~~~ xxxx项目~~~~(编号xxx-xxx-xxx)
%\end{publist}
%
%{\hei 学术论文}
%\begin{publist}
%\item 在~xxxxxxx~等刊物发表论文多篇
%\item 在~xxxxxxxxxxxxxxxx~等多个国际会议上发表论文多篇
%\end{publist}
%
          % 个人简历

\clearpage
\ifx\atempxetex\usewhat\else
\end{CJK*}
\fi

\end{document}
