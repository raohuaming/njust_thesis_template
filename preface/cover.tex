% !Mode:: "TeX:UTF-8"
% 若xelatex编译非UTF8文件,需在每个文件中指定字符编码;
% main.tex中手动制定了\atemp和\usewhat参数;
\ifx\atempxetex\usewhat
%\XeTeXinputencoding "gbk"
\fi

\newcommand{\chinesethesistitle}{xxxxxxxxxxxxxxxx} %授权书用,无需断行
\newcommand{\englishthesistitle}{xxxxxxxxxxxxxxxxg}
\newcommand{\chinesethesistime}{2013~年~6月}  %封面底部的日期中文形式
\newcommand{\dchinesethesistime}{2013~年~6月} %答辩日期中文形式:2011~年~~月
\newcommand{\englishthesistime}{June, 2013}   %封面底部的日期英文形式
%-----定义标签值-与format.tex文件中的定义相对应----------------
\lnatclassifiedindex{分类号}
\lsecretclassifiedindex{密级}
\linternatclassifiedindex{UDC}

\lsupervisor{指导教师}
\lname{姓名}
\lauthor{作\hspace*{2em}者}
\ldegree{申请学位级别}
\lsubject{专业名称}
\ldate{论文提交日期}
\lddate{论文答辩日期}
\lschool{学位授予单位和日期}
\lchairman{答辩委员会主席}
\lexpositor{评阅人}
\cabstract{摘\hspace*{2em}要}
\ckeywords{关键词}

%----定义标签对应选项的值---------------
\natclassifiedindex{TP309}
\secretclassifiedindex{公开}
\internatclassifiedindex{~681.324} %国际图书分类号

%\school{南京理工大学}
%\cdegree{\cxuewei}
\ctitle{\erhao\hei xxxxxxxxxxxxxxxx}  %封面用论文标题,自己可手动断行
\csubject{xxxxxxxxxxxxxxxx}                 %(~按二级学科填写~)
\cauthor{xxxx}            %作者姓名:XXX
\crole{教授}         %导师职称:教授
\csupervisor{XXXX} %导师名字:XXXX
%\cassosupervisor{某~~~~~~某~~~~教~~授}     %(~如无副导师可以不列此项~)
%\ccosupervisor{某~~某~~某~~~~教~~授~} %(~如无联合培养导师则不列此项~)
\cdate{\chinesethesistime}
\ddate{\dchinesethesistime}%定义答辩日期

\etitle{\englishthesistitle}
\edegree{\exuewei \ of \exueke}
\esubject{Pattern Recognition and Intelligence Systems}  %英文二级学科名
\eaffil{School of Computer Science and Technology}%英文单位 %换行用\newline,不要用\\
\eauthor{\textbf{XXXX}}                   %作者姓名 (英文):\textbf{Yantingt Lu}
\esupervisor{\textbf{XXXX}}       % 导师姓名 (英文):\textbf{Jing-Yu Yang}
%\ecosupervisor{Professor X}
%\eassosupervisor{Professor Y}
\edate{\englishthesistime}


\cabstract{
XXXXXXXXXXXXXXXXXXXXXXXXXXXXXXXXXXX

XXXXXXXXXXXXXXXXXXXXXXXXXXXXXXXXXXX

}

%关键词不能放在摘要页的外面
 \ckeywords{XXXXX,\ XXXXX,\ XXXXX, \ XXXXX, \ XXXXX, \ XXXXX, \ XXXXX}



\eabstract{
XXXXXXXXXXXXXXXXXXXXXXXXXXXXXXXXXXX

XXXXXXXXXXXXXXXXXXXXXXXXXXXXXXXXXXX

}

\ekeywords{XXXXX, \ XXXXX, \ XXXXX, \ XXXXX,  \ XXXXX,  \ XXXXX, \ XXXXX}

%主要符号说明
%\NotationList{
%%\vspace*{10pt}%可在主要符号说明与下面内容之间留空
%\begin{center}
%\begin{tabular}{lll}
%符号 & 名字 & 定义\\
%\hline
%$\mathcal {X}$,$\mathcal {Y}$,$\mathcal {Z}$ & 花体大写英文& 张量\\
%$\textbf{A}$, $\textbf{B}$, $\textbf{C}$ & 粗体大写英文& 矩阵 \\
%$\textbf{u}$, $\textbf{v}$, $\textbf{w}$ & 粗体小写英文& 向量 \\
%$l, m, n$ & 常规斜体小写英文& 数量 \\
%$L, M, N$ & 常规斜体大写英文& 数量指标的上界(如$m=1,\ldots,M$) \\
%$\alpha ,\beta ,\theta $ & 小写希腊字母 & 参数 \\
%$I_{n}$ & 专用符号符号 & $n\times n$单位矩阵 \\
%$1_{m\times n}$ & 专用符号 & $m\times n$全一矩阵 \\
%$\epsilon$ & 专用符号 & 误差 \\
%$\lambda $ & 专用符号 & 矩阵的特征值 \\
%$\otimes  $ & 专用符号 & Kronecker积 \\
%$\langle \bullet , \bullet \rangle$ & 专用符号 & 向量内积 \\
%\hline
%\end{tabular}
%\end{center}
%}
\makecover
\clearpage
